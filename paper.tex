%%%%%%%%%%%%%%%%%%%%%%%%%%%%%%%%%%%%%%%%%%%%%%%%%%%%%%%
% % Lines starting with % are comments, which are ignored.
% % This is a handy way of indicating the date and version of
% % your document, to wit:
% %
% % LaTeX sample file
% % Modified March, 2002
% %
%%%%%%%%%%%%%%%%%%%%%%%%%%%%%%%%%%%%%%%%%%%%%%%%%%%%%%%
% % Title and author(s)
%%%%%%%%%%%%%%%%%%%%%%%%%%%%%%%%%%%%%%%%%%%%%%%%%%%%%%%
\title{Fourier Series}
\author{Jacob Bieker}
%%%%%%%%%%%%%%%%%%%%%%%%%%%%%%%%%%%%%%%%%%%%%%%%%%%%%%%
\documentclass{article}
%%%%%%%%%%%%%%%%%%%%%%%%%%%%%%%%%%%%%%%%%%%%%%%%%%%%%%%
% %
% % The next command allows your in import encapsulated
% % postscript files, .epsf or .eps files, which
% % contain vector graphic image data.
% %
%%%%%%%%%%%%%%%%%%%%%%%%%%%%%%%%%%%%%%%%%%%%%%%%%%%%%%%
\usepackage{graphicx}
%%%%%%%%%%%%%%%%%%%%%%%%%%%%%%%%%%%%%%%%%%%%%%%%%%%%%%%
% % We use newtheorem to define theorem-like structures
% %
% % Here are some common ones. . .
%%%%%%%%%%%%%%%%%%%%%%%%%%%%%%%%%%%%%%%%%%%%%%%%%%%%%%%
\newtheorem{theorem}{Theorem}
\newtheorem{lemma}{Lemma}
\newtheorem{proposition}{Proposition}
\newtheorem{scolium}{Scolium}   %% And a not so common one.
\newtheorem{definition}{Definition}
\newenvironment{proof}{{\sc Proof:}}{~\hfill QED}
\newenvironment{AMS}{}{}
\newenvironment{keywords}{}{}
\renewcommand{\baselinestretch}{1.6} 
%%%%%%%%%%%%%%%%%%%%%%%%%%%%%%%%%%%%%%%%%%%%%%%%%%%%%%%
% %   The first thanks indicates your affiliation
% %
% %  Just the name here.
% %
% % Your mailing address goes at the end.
% %
% % \thanks is also how you indicate grant support
% %
%%%%%%%%%%%%%%%%%%%%%%%%%%%%%%%%%%%%%%%%%%%%%%%%%%%%%%%


\begin{document}
\newpage
\maketitle
%%%%%%%%%%%%%%%%%%%%%%%%%%%
% abstract, keywords and Subject classification are optional.
%%%%%%%%%%%%%%%%%%%%%%%%%%%

% Most people don't use these, so they are "commented out"
% by starting the lines with a "%"
%\begin{keywords}
%   \LaTeX, typesetting
%\end{keywords}

%\begin{AMS}
%   50C60, 18C25
%\end{AMS}

%%%%%%%%%%%%%%%%%%%%%%
% % Here is the start of the Text
%%%%%%%%%%%%%%%%%%%%%%
\section{Description}
Fourier series are expansions of periodic functions in terms of an infinite sum of sines and cosines.
It is a way to describe functions in a combination of simpler functions and
a way to make analysis of physical systems easier.

Fourier Series generally take the form of 

$$s(x) = \frac{a_0}{2} + \sum_{n=1}^\infty \left[a_n\cos\left(nx\right)+b_n\sin\left(nx\right)\right]$$

For odd functions $a_n$ becomes zero and the Fourier series becomes composed of only sine functions, while if the function is even, $b_n$ becomes zero and the Fourier series becomes composed of only cosine functions. 

In general,
$$a_n {} = \frac{1}{\pi}\int_{-\pi}^{\pi}s(x) \cos(nx)\,dx$$
and
$$b_n {} = \frac{1}{\pi}\int_{-\pi}^{\pi}s(x) \sin(nx)\, dx$$
where $\quad n \ge 0$, and $s(x)$ is the function being described by the Fourier Series.\cite{WikiFourier}
\section{Methods of Studying Fourier Series}\label{Description}
There are multiple methods to studying Fourier Series, including the Fast Fourier Transform, Discrete-Time Fourier Transform, the Discrete Fourier Transform, and convolving the transformations.
     
\subsection{Fast Fourier Transformations}
The Fast Fourier Transform is a way of transforming a function of time into a function of frequency. It is useful when studying time-dependent phenomena. In comparison to the Discrete Fourier Transform described below, which takes $O(n^2)$ computational steps, the Fast Fourier Transform can be computed in $O(nlog(n))$ time, a considerable speedup when $n$ becomes large.\cite{CarnegieFFT}

There are multiple different Fast Fourier Transform algorithms, including the Cooley-Tukey algorithm, among others. The Cooley-Tukey algorithm, named after J.W. Cooley of IBM and John Tukey of Princeton, breaks the Discrete Fourier Transform into smaller Discrete Fourier Transforms of arbitrary size recursively.\cite{CarnegieFFT}
\subsection{Discrete-Time Fourier Transform}
\begin{definition}
	The {\em Discrete-time Fourier Transform (DTFT)} can be defined as:
	$$X_{2\pi}(\omega) = \sum_{n=-\infty}^{\infty} x[n] \,e^{-i \omega n}$$
	where $x[n]$ is the set of all discrete real and complex numbers for integer $n$, and $\omega$ is the frequency variable with normalized units of radians/sample.\cite{UBC}
\end{definition}

Discrete-time Fourier Transforms are useful in that its inverse is the original data sequence.
\subsection{Discrete Fourier Transformation}
\begin{definition}
	Given $n$ real or complex inputs $x_{0}...x_{n-1}$, the Discrete Fourier Transform is defined as $$y_{k} = \sum_{0\leq l < n} \omega_n^{kl}x_l, 0 \leq k < n $$, 
	where $\omega_n = e^{\frac{-2i\pi}{n}}$.\cite{CarnegieFFT}
\end{definition}
The Discrete Fourier Transform differs from the Discrete-time Fourier Transform in that both the input and the output functions are both finite.


\subsection{Convolution}
\begin{theorem}
	The Convolution Theorem states that, under suitable conditions, the Fourier transform of a convolution is equal to the pointwise product of Fourier transforms.\cite{BerkConvolution}\cite{Convolution}
\end{theorem}

For example, if $f$ and $g$ are two different functions with the convolution of $f*g$, and the Fourier transforms of $f$ and $g$ are defined as $\mathcal{F}(f)$ and $\mathcal{F}(g)$ respectively, then the following is true

$$\mathcal{F}\{f*g\} = \mathcal{F}\{f\} \cdot \mathcal{F}\{g\}$$

and so is 

$$\mathcal{F}\{f\} \cdot \mathcal{F}\{g\} = \mathcal{F}\{f*g\}$$

The caveat is that, depending on how the Fourier transforms are normalized, there might be a constant scaling factor that appears in the above. \cite{BerkConvolution}

The Convolution Theorem is exceedingly useful in creating programs and computers to solve convolutions, as the time complexity can be reduced from $O(n^4)$ to $O(nlogn)$.\cite{Convolution}

\section{Applications of Fourier Series}
It is used in proving the Nyquist-Shannon sampling theorem,
studying harmonic oscillations, waveforms, signal processing,
diffractions, interference, and Young's double slit experiment,
and radiation from surface currents, supernovae simulations,
partial differential equations by separating variables,
solving the heat equation.

\subsection{Nyquist-Shannon Sampling Theorem}
\begin{theorem}
	If a time-varying signal is periodically sampled at a rate of at least twice the frequency of the highest sinusoidal component contained within the signal, then the original time-varying signal can be exactly recovered from the periodic samples.\cite{Nyquist}
\end{theorem}
The theorem explains why aliasing occurs when the sampling rate is too low, so that the sample's higher frequencies are not properly accounted for and instead masquerade as a different frequency.
\subsection{Signal Processing}
In signal processing, the Fourier series, and especially the Fourier transform, is used to determine what frequencies are present in a signal, and in what proportions. The magnitude squared of the Fourier series of a given signal gives the amount of power the signal has at that particular frequency. The Fourier series also makes it easier for specific frequencies present in a signal to be blocked out, or nullified.\cite{FourierHeatSignal}\cite{WikiFourier}
\subsection{Astrophysical Simulations}
As part of research I am conducting with Dr. James Imamura, where we simulate stars as they accumulate mass and become unstable, the Fourier series coefficients are graphed and examined to see how the simulation is going. As the star becomes more unstable, the first few terms grow larger and fluctuate more dramatically.

It is also used in the calculation of Rossby Wave Instabilities. The instabilities appear in rotating disks and are potentially one avenue of planetary formation. Fourier series come into play when the instabilities are simulated, with the self-gravity of the system calculated using a Fourier transform.\cite{Lovelace}
\subsection{Heat Equation}
Solving the heat equation was one of the first applications of the Fourier series, and is what Joseph Fourier proposed when first coming up with the Fourier series.\cite{FourierHeatSignal}

\subsection{Quantum Physics}
In quantum physics, the Fourier transform can be used with wavefunctions, as well as with the uncertainty principle to show that completely knowing the momentum of a particle means that the location is completely unknown, as well as the opposite. For example, by describing the physical state of a particle, such as its momentum $p$, and position, $q$, the set of possible states of the particle can be described by the Fourier transform of 

$$\phi(p) = \int \psi (q) e^{2\pi i pq\over h} dq$$

This Fourier transformation allows for changing between different ways of representing the state of a particle, such as between momentum and position, in a simple manner.\cite{WikiTransform}

The Fourier transform is equivalent to splitting light into its component spectrum. It is also used to understand phenomena in optics and with light, such as Young's double slit experiment. For the double slit experiment, the Fourier transform of the transmission function gives the frequency content of the resultant wave, while the square of the Fourier transform gives the intensity of the pattern on the screen. The transform then describes the maxima and minima that occur from the double slit experiment correctly.\cite{Optics}

\begin{thebibliography}{9}

	
	\bibitem{Lovelace} Lovelace, R.~V.~E., \& Romanova, M.~M.\ 2014, Fluid Dynamics Research, 46, 041401 
	
	\bibitem{UBC}Feldman, Joel. "Discrete-time Fourier Transform and Fourier Series." University of British Columbia, 21 Mar. 2007. Web. 2 Dec. 2015.

	\bibitem{Nyquist}"Nyquist Sampling Theorem." Portland Community College, n.d. Web. 2 Dec. 2015.

	\bibitem{Aliasing}Olshausen, Bruno A. "Aliasing." Aliasing. Berkeley, 10 Oct. 2000. Web. 2 Dec. 2015.

	\bibitem{Optics}"Optics And Fourier Transform." Optics. Berkeley, n.d. Web. 02 Dec. 2015.
	
	\bibitem{FourierHeatSignal}Jackson, Leland B. "Sampling and Fourier Analysis." Digital Filters and Signal Processing (1996): 139-87. Web. 2 Dec. 2015.
	
	\bibitem{BerkConvolution}Graham, James R. "Distribution Theory Convolution, Fourier Transform, and Laplace Transform." (2013): n. pag. Berkeley. Web. 2 Dec. 2015.
	
	\bibitem{CarnegieFFT}Franchetti, Franz, and Markus P ̈uschel. "Fast Fourier Transform." The Concord Saunterer 18.1 (1985): 38-50. Carnegie Mellon University. Web. 2 Dec. 2015.
	
	\bibitem{Convolution}"Convolution Theorem." Wikipedia. Wikimedia Foundation, n.d. Web. 02 Dec. 2015.
	
	\bibitem{WikiDTFT}"Discrete Time Fourier Transform." Wikipedia. Wikimedia Foundation, n.d. Web. 02 Dec. 2015.
	
	\bibitem{WikiFourier}"Fourier Analysis." Wikipedia. Wikimedia Foundation, n.d. Web. 02 Dec. 2015.
	
	\bibitem{WikiTransform}"Fourier Transform." Wikipedia. Wikimedia Foundation, n.d. Web. 02 Dec. 2015.
\end{thebibliography}

\end{document}
